\documentclass{beamer}
%\usepackage[latin1]{inputenc}
\usepackage{graphicx}
\usepackage{Images/sas}
\usepackage{pbox}
\usepackage{verbatim}
\usetheme{Warsaw}

\title[BACI]{Design and Analysis of BACI Experiments.
}
\author{Carl James Schwarz}\institute{Department of Statistics and Actuarial Science \\ Simon Fraser  University \\ Burnaby, BC, Canada \\ cschwarz @ stat.sfu.ca}

%\date{}


% refer to the following URL for hints on beamer
%        http://www.math.umbc.edu/~rouben/beamer/quickstart-Z-H-27.html#node_sec_27
%gets rid of bottom navigation bars
\setbeamertemplate{footline}[page number]{}

%gets rid of navigation symbols
\setbeamertemplate{navigation symbols}{}

% set background color to light shade of blude
\setbeamercolor{normal text}{bg=blue!12}

% To create 2 x3 handouts 
% use the Preview program in Macintosh. You can set up the number of slides/page with the layout options 
% in the usual fashion.
% Note that Adobe will not let you save a pdf file to another pdf file using the print -> pdf options - you must use Preview



\begin{document}

%******************************************************************************************************************
\begin{frame}
\titlepage
\end{frame}


%******************************************************************************************************************
\begin{frame}
\frametitle{Abstract}

Before-After-Control-Impact (BACI) designs are commonly used to monitor for 
potential environmental impacts.  We will review the four most common BACI 
designs highlighting the proper analyses of these designs, their limitations in the 
interpretation of the results (and alternatives), and what information is required for planning purposes. 
Finally, we will review alternatives to BACI designs, especially when control information is limited or lacking. 

\vspace{4mm}
Additional information is available at:
\begin{itemize}
\item \url{http://www.stat.sfu.ca/~cschwarz/CourseNotes}
\item Follow link on above page to {\it Sample Program Library} for
{\it SAS}, {\it JMP}, and {\it R} code.
\end{itemize}
\end{frame}


%******************************************************************************************************************
\begin{frame}
\frametitle{Why BACI?}

\begin{itemize}
\item Temporal changes may be confounded with environmental impact.
\item Site-differences may be unrelated to environmental impact.
\end{itemize}
\includegraphics[height=0.6 \textheight]{Images/BACI-conceptual}\\
{\bf Non-parallelism in response = Environmental Impact}
\end{frame}

%******************************************************************************************************************
\begin{frame}
\frametitle{Key BACI (hidden) assumption}

\includegraphics[height=0.8 \textheight]{Images/BACI-key-assumption}\\

{\center Step change between Before and After Impact!}
\end{frame}

%******************************************************************************************************************
\begin{frame}
\frametitle{When is BACI best?}

BACI designs are good for:
\begin{itemize}
\item Large potential changes after impact
\item Changes are permanent after impact
\item Monitoring to protect against disasters
\item Monitoring for changes in the MEAN
\end{itemize}
 \vspace{1cm}
 
BACI designs are poor for:
\begin{itemize}
\item Small potential changes after impact
\item Gradual changes after impact (i.e.\ not a step change)
\item Long term monitoring
\item Monitoring for changes in VARIABILITY
\end{itemize}

\end{frame}


%******************************************************************************************************************
\begin{frame}
\frametitle{Four standard BACI Designs}
Four standard BACI designs:
\begin{enumerate}
\item Single impact site; single control site; one year before; one year after.
\item Single/multiple impact site; multiple control sites; one year before; one year after.
\item Single impact site; single control site; multiple years before; multiple years after.
\item Single impact site; multiple control sites; multiple years before; multiple years after.
\end{enumerate}

\end{frame}

%******************************************************************************************************************
\begin{frame}
\frametitle{BACI Design 1}

\includegraphics[height=0.7 \textheight]{Images/BACI-design-1}\\
E.g.\ Number of shore crabs on beaches affected by cooling water of power plant measured using independent quadrats each year.

\end{frame}


%******************************************************************************************************************
\begin{frame}
\frametitle{BACI Design 1 - Decomposition of Effects}

\includegraphics[height=0.7 \textheight]{Images/BACI-design-1-decomposition1.png}
\end{frame}

%******************************************************************************************************************
\begin{frame}
\frametitle{BACI Design 1 - Decomposition of Effects}

\includegraphics[height=0.7 \textheight]{Images/BACI-design-1-decomposition2.png}
\end{frame}

%******************************************************************************************************************
\begin{frame}
\frametitle{BACI Design 1 - Decomposition of Effects}

\includegraphics[height=0.7 \textheight]{Images/BACI-design-1-decomposition3.png}
\end{frame}



%******************************************************************************************************************
\begin{frame}
\frametitle{BACI Design 1}
Two-factor completely-randomized design ANOVA.\\
\vspace{3mm}

{\it SAS} Analysis:\\
\hspace{1cm} PROC Mixed data=blah;\\
\hspace{1cm} CLASS Period Siteclass;\\
\hspace{1cm} MODEL Y = Period~~ Siteclass~~ Period*Siteclass;\\
\hspace{1cm} ESTIMATE 'BACI Effect' Period*Siteclass 1 -1 -1 1;\\

\vspace{3mm}
The test of no impact is the test for no interaction, i.e.\ test of $Period*Siteclass$ interaction.\\
Estimate BACI effect as DIFFERENTIAL change in means between two sites, i.e.\\[-4mm]
$$BACI=(\mu_{ca}-\mu_{cb}) - (\mu_{ta}-\mu_{tb})$$
\end{frame}


%******************************************************************************************************************
\begin{frame}
\frametitle{BACI Design 1}
Two-factor completely-randomized design ANOVA.\\

\vspace{3mm}
{\it SAS} results (with 4/5/6/4 replicates at site-year combinations):\\[-4mm]
\sascontents[1]{The Mixed Procedure}
\sascontents[2]{Type 3 Tests of Fixed Effects}


\begin{sastable}[c]{lrrrr}\hline
   \multicolumn{5}{|S{header}{c}|}{\pbox[b]{\textwidth}{Type 3 Tests of Fixed Effects}}
\\\hline
   \multicolumn{1}{|S{header}{l}|}{\pbox[b]{\textwidth}{Effect}} & 
   \multicolumn{1}{|S{header}{r}|}{\pbox[b]{\textwidth}{Num DF}} & 
   \multicolumn{1}{|S{header}{r}|}{\pbox[b]{\textwidth}{Den DF}} & 
   \multicolumn{1}{|S{header}{r}|}{\pbox[b]{\textwidth}{F Value}} & 
   \multicolumn{1}{|S{header}{r}|}{\pbox[b]{\textwidth}{Pr~>~F}}
\\\hline
\endhead
   \multicolumn{1}{|S{rowheader}{l}|}{\pbox[b]{\textwidth}{SiteClass}} & 
   \multicolumn{1}{|S{data}{r}|}{\pbox[b]{\textwidth}{1}} & 
   \multicolumn{1}{|S{data}{r}|}{\pbox[b]{\textwidth}{15}} & 
   \multicolumn{1}{|S{data}{r}|}{\pbox[b]{\textwidth}{13.90}} & 
   \multicolumn{1}{|S{data}{r}|}{\pbox[b]{\textwidth}{0.0020}}
\\\hline
   \multicolumn{1}{|S{rowheader}{l}|}{\pbox[b]{\textwidth}{Period}} & 
   \multicolumn{1}{|S{data}{r}|}{\pbox[b]{\textwidth}{1}} & 
   \multicolumn{1}{|S{data}{r}|}{\pbox[b]{\textwidth}{15}} & 
   \multicolumn{1}{|S{data}{r}|}{\pbox[b]{\textwidth}{8.54}} & 
   \multicolumn{1}{|S{data}{r}|}{\pbox[b]{\textwidth}{0.0105}}
\\\hline
   \multicolumn{1}{|S{rowheader}{l}|}{\pbox[b]{\textwidth}{SiteClass*Period}} & 
   \multicolumn{1}{|S{data}{r}|}{\pbox[b]{\textwidth}{1}} & 
   \multicolumn{1}{|S{data}{r}|}{\pbox[b]{\textwidth}{15}} & 
   \multicolumn{1}{|S{data}{r}|}{\pbox[b]{\textwidth}{0.97}} & 
   \multicolumn{1}{|S{data}{r}|}{\pbox[b]{\textwidth}{0.3404}}
\\\hline
\end{sastable}




\sascontents[1]{The Print Procedure}
\sascontents[2]{Data Set WORK.MIXEDESTS}


\begin{sastable}[c]{lrrrrrrrr}\hline
   \multicolumn{1}{|S{header}{l}|}{\pbox[b]{\textwidth}{Label}} & 
   \multicolumn{1}{|S{header}{r}|}{\pbox[b]{\textwidth}{Estimate}} & 
   \multicolumn{1}{|S{header}{r}|}{\pbox[b]{\textwidth}{Standard \\ Error}} & 
   \multicolumn{1}{|S{header}{r}|}{\pbox[b]{\textwidth}{DF}} & 
   \multicolumn{1}{|S{header}{r}|}{\pbox[b]{\textwidth}{t \\ Value}} & 
   \multicolumn{1}{|S{header}{r}|}{\pbox[b]{\textwidth}{Pr \\ > \\ |t|}} & 
   \multicolumn{1}{|S{header}{r}|}{\pbox[b]{\textwidth}{Alpha}} & 
   \multicolumn{1}{|S{header}{r}|}{\pbox[b]{\textwidth}{Lower}} & 
   \multicolumn{1}{|S{header}{r}|}{\pbox[b]{\textwidth}{Upper}}
\\\hline
\endhead
   \multicolumn{1}{|S{data}{l}|}{\pbox[b]{\textwidth}{BACI effect}} & 
   \multicolumn{1}{|S{data}{r}|}{\pbox[b]{\textwidth}{  3.0500}} & 
   \multicolumn{1}{|S{data}{r}|}{\pbox[b]{\textwidth}{  3.0974}} & 
   \multicolumn{1}{|S{data}{r}|}{\pbox[b]{\textwidth}{  15}} & 
   \multicolumn{1}{|S{data}{r}|}{\pbox[b]{\textwidth}{   0.98}} & 
   \multicolumn{1}{|S{data}{r}|}{\pbox[b]{\textwidth}{0.3404}} & 
   \multicolumn{1}{|S{data}{r}|}{\pbox[b]{\textwidth}{  0.05}} & 
   \multicolumn{1}{|S{data}{r}|}{\pbox[b]{\textwidth}{ -3.5520}} & 
   \multicolumn{1}{|S{data}{r}|}{\pbox[b]{\textwidth}{  9.6520}}
\\\hline
\end{sastable}


\end{frame}

%******************************************************************************************************************
\begin{frame}
\frametitle{BACI Design 1}

Two-factor completely-randomized design ANOVA.\\
\vspace{3mm}

{\it R} Analysis: (Be sure that Period and Siteclass are FACTORS).\\
\vspace{2mm}
\hspace{5mm} results<-lm(Y$\sim$Period+ Siteclass+Period:Siteclass, data=blah)\\
\hspace{5mm} anova(results)\\
\hspace{5mm} summary(results)\\

\vspace{2mm}
CAUTION: In {\it R}, you must put the interaction term last if design is unbalanced otherwise the test for interaction is NOT correct.


\end{frame}


%******************************************************************************************************************
\begin{frame}[fragile]  % must declare fragile if using \verbatim in the slide
\frametitle{BACI Design 1}
Two-factor completely-randomized design ANOVA.\\

\vspace{3mm}
{\it R} results:\\[-8mm]
\begin{verbatim}
Response: Density
                 Df  Sum Sq Mean Sq F value    Pr(>F)    
SiteClass         1 194.695 194.695 17.5877 0.0007827 ***  WRONG
Period            1  92.205  92.205  8.3293 0.0113100 *    WRONG 
SiteClass:Period  1  10.734  10.734  0.9696 0.3403935    
Residuals        15 166.050  11.070                      
\end{verbatim}

\vspace{2mm} 
Note that tests for {\it SiteClass} and {\it Period} are MISLEADING 
and don't test what you think they do!

\vspace{2mm} 
Estimate of BACI effect is:\\[-8mm]
\verbatiminput{Images/baci-crabs-R-100baci.txt}

\end{frame}


%******************************************************************************************************************
\begin{frame}
\frametitle{BACI Design 1}
Assumptions:
\begin{itemize}
\item Not necessary for design to be balanced, i.e.\ number of replicates can vary across site-time.
\item All measurements within and across years at a site are INDEPENDENT of each other
\item Normality of residuals (but fairly robust IF....)
\item Equality of variation each each Site $\times$ Period
\end{itemize}

\vspace{2mm}
Key Limitations:
\begin{itemize}
\item Effect may be an artifact of sites chosen or years chosen.
\item Inference is LIMITED to those particular sites and years chosen.
\end{itemize}

\end{frame}


%******************************************************************************************************************
\begin{frame}
\frametitle{BACI Design 1}

\includegraphics[height=0.5 \textheight]{Images/BACI-design-1}\\

Power (see web site for programs) depends on:
\begin{itemize}
\item Size of BACI effect $=(\mu_{ca}-\mu_{cb}) - (\mu_{ta}-\mu_{tb})$
\item Std DEV ($\sigma$) of replicates at each site-year combination
\item Number of replicates at each site-year combination
\end{itemize}

\end{frame}


%******************************************************************************************************************
\begin{frame}
\frametitle{BACI Design 1}

Crabs example: $BACI=-5$, $\sigma=3.32$ 

\verbatiminput{Images/baci-crabs-power-R-500.txt}

\vspace{2mm}
Aim for about 80\% power at $\alpha=0.05$.
\end{frame}





%%%%%%%%%%%%%%%%%%%%%%%%%%%%%%%%%%%%%

%******************************************************************************************************************
\begin{frame}
\frametitle{BACI Design 2}
\includegraphics[height=0.7 \textheight]{Images/BACI-design-2-multicontrols}

E.g.\ Number of shore crabs on beaches affected by cooling water of power plant measured using independent quadrats each year with multiple beaches for controls.

\end{frame}


%******************************************************************************************************************
\begin{frame}
\frametitle{BACI Design 2}
\includegraphics[height=0.4 \textheight]{Images/BACI-design-2-multicontrols-sourcevar.png}

Three levels of variation: 
\begin{itemize}
\item  quadrats-within-a-beach; These are pseudo-replicates. Variability can be reduced by increasing the size of the quadrats.
\item site-to-site. Site serve a ``blocks'' so main effect site-differences are NOT important.
\item Site-Year. This represents the inconsistent temporal effects over sites.
\end{itemize}
Note that if you only have one quadrat, then quadrat and site*year variation is confounded and cannot be separated.
\end{frame}



\begin{frame}
\frametitle{BACI Design 2}
\includegraphics[height=0.4 \textheight]{Images/BACI-design-2-multicontrols-sourcevar.png}\\
What can you control via modifications to the sampling design?
\begin{itemize}
\item Quadrat variation can be controlled by choosing larger quadrats.
\item Site variation cannot be controlled, but by measuring all sites in all years, site effects ``cancel''.
\item Site-Year variation cannot be controlled by modifying design. This is the limiting variation for the design.
\end{itemize}
\end{frame}


%******************************************************************************************************************
\begin{frame}
\frametitle{BACI Design 2}
Multiple (equivalent) ways to analyze this data:
\begin{itemize}
\item Find difference of means for EACH site; analyze the differences in means using $t$-test to see if the mean difference among the controls = mean difference for impact site. [Only approximate analysis if design is unbalanced.]
\item Mixed Effects ANOVA on the means of each site-year combination.
\item Mixed Effects ANOVA on the raw data. This is the most general and provides estimates of ALL variance components needed for power analysis.
\end{itemize}

\end{frame}


%******************************************************************************************************************
\begin{frame}
\frametitle{BACI Design 2}
Two-factor mixed-effect ANOVA.\\

{\it SAS} Analysis:\\
\hspace{5mm} PROC Mixed data=blah;\\
\hspace{5mm} CLASS Period Siteclass Site;\\
\hspace{5mm} MODEL Y = Period~~ Siteclass~~ Period*Siteclass / ddfm=kr;\\
\hspace{5mm} RANDOM site(SiteClass) site(Siteclass)*Period;\\
\hspace{5mm} ESTIMATE 'BACI Effect' Period*Siteclass 1 -1 -1 1;\\

\vspace{3mm}
The test of no impact is the test for no interaction, i.e.\ test of $Period*Siteclass$ interaction.\\
Estimate BACI effect as DIFFERENTIAL change in means between two sites, i.e.\\[-4mm]
$$BACI=(\mu_{ca}-\mu_{cb}) - (\mu_{ta}-\mu_{tb})$$
\end{frame}


%******************************************************************************************************************
\begin{frame}
\frametitle{BACI Design 2}
Two-factor mixed-effect ANOVA.\\

\vspace{3mm}
{\it SAS} results (2C, 1I site; 30 quadrats in total):\\[-4mm]
\sascontents[1]{The Mixed Procedure}
\sascontents[2]{Type 3 Tests of Fixed Effects}


\begin{sastable}[c]{lrrrr}\hline
   \multicolumn{5}{|S{header}{c}|}{\pbox[b]{\textwidth}{Type 3 Tests of Fixed Effects}}
\\\hline
   \multicolumn{1}{|S{header}{l}|}{\pbox[b]{\textwidth}{Effect}} & 
   \multicolumn{1}{|S{header}{r}|}{\pbox[b]{\textwidth}{Num DF}} & 
   \multicolumn{1}{|S{header}{r}|}{\pbox[b]{\textwidth}{Den DF}} & 
   \multicolumn{1}{|S{header}{r}|}{\pbox[b]{\textwidth}{F Value}} & 
   \multicolumn{1}{|S{header}{r}|}{\pbox[b]{\textwidth}{Pr~>~F}}
\\\hline
\endhead
   \multicolumn{1}{|S{rowheader}{l}|}{\pbox[b]{\textwidth}{SiteClass}} & 
   \multicolumn{1}{|S{data}{r}|}{\pbox[b]{\textwidth}{1}} & 
   \multicolumn{1}{|S{data}{r}|}{\pbox[b]{\textwidth}{1.01}} & 
   \multicolumn{1}{|S{data}{r}|}{\pbox[b]{\textwidth}{2.92}} & 
   \multicolumn{1}{|S{data}{r}|}{\pbox[b]{\textwidth}{0.3351}}
\\\hline
   \multicolumn{1}{|S{rowheader}{l}|}{\pbox[b]{\textwidth}{Period}} & 
   \multicolumn{1}{|S{data}{r}|}{\pbox[b]{\textwidth}{1}} & 
   \multicolumn{1}{|S{data}{r}|}{\pbox[b]{\textwidth}{1.11}} & 
   \multicolumn{1}{|S{data}{r}|}{\pbox[b]{\textwidth}{9.99}} & 
   \multicolumn{1}{|S{data}{r}|}{\pbox[b]{\textwidth}{0.1746}}
\\\hline
   \multicolumn{1}{|S{rowheader}{l}|}{\pbox[b]{\textwidth}{SiteClass*Period}} & 
   \multicolumn{1}{|S{data}{r}|}{\pbox[b]{\textwidth}{1}} & 
   \multicolumn{1}{|S{data}{r}|}{\pbox[b]{\textwidth}{1.11}} & 
   \multicolumn{1}{|S{data}{r}|}{\pbox[b]{\textwidth}{0.10}} & 
   \multicolumn{1}{|S{data}{r}|}{\pbox[b]{\textwidth}{0.8002}}
\\\hline
\end{sastable}



\sascontents[1]{The Print Procedure}
\sascontents[2]{Data Set WORK.MIXED2ESTS}


\begin{sastable}[c]{lrrrrrrrr}\hline
   \multicolumn{1}{|S{header}{l}|}{\pbox[b]{\textwidth}{Label}} & 
   \multicolumn{1}{|S{header}{r}|}{\pbox[b]{\textwidth}{Estimate}} & 
   \multicolumn{1}{|S{header}{r}|}{\pbox[b]{\textwidth}{Standard \\ Error}} & 
   \multicolumn{1}{|S{header}{r}|}{\pbox[b]{\textwidth}{DF}} & 
   \multicolumn{1}{|S{header}{r}|}{\pbox[b]{\textwidth}{t \\ Value}} & 
   \multicolumn{1}{|S{header}{r}|}{\pbox[b]{\textwidth}{Pr \\ > \\ |t|}} & 
   \multicolumn{1}{|S{header}{r}|}{\pbox[b]{\textwidth}{Alpha}} & 
   \multicolumn{1}{|S{header}{r}|}{\pbox[b]{\textwidth}{Lower}} & 
   \multicolumn{1}{|S{header}{r}|}{\pbox[b]{\textwidth}{Upper}}
\\\hline
\endhead
   \multicolumn{1}{|S{data}{l}|}{\pbox[b]{\textwidth}{BACI effect}} & 
   \multicolumn{1}{|S{data}{r}|}{\pbox[b]{\textwidth}{  1.1038}} & 
   \multicolumn{1}{|S{data}{r}|}{\pbox[b]{\textwidth}{  3.4794}} & 
   \multicolumn{1}{|S{data}{r}|}{\pbox[b]{\textwidth}{1.11}} & 
   \multicolumn{1}{|S{data}{r}|}{\pbox[b]{\textwidth}{   0.32}} & 
   \multicolumn{1}{|S{data}{r}|}{\pbox[b]{\textwidth}{0.8002}} & 
   \multicolumn{1}{|S{data}{r}|}{\pbox[b]{\textwidth}{  0.05}} & 
   \multicolumn{1}{|S{data}{r}|}{\pbox[b]{\textwidth}{-33.8575}} & 
   \multicolumn{1}{|S{data}{r}|}{\pbox[b]{\textwidth}{ 36.0650}}
\\\hline
\end{sastable}


\end{frame}

%******************************************************************************************************************
\begin{frame}
\frametitle{BACI Design 2}
Two-factor mixed-effect ANOVA.\\

\vspace{3mm}
{\it SAS} results (continued):\\[-4mm]

\sascontents[1]{The Print Procedure}
\sascontents[2]{Data Set WORK.MIXED2COVPARMS}


\begin{sastable}[c]{lr}\hline
   \multicolumn{1}{|S{header}{l}|}{\pbox[b]{\textwidth}{Cov \\ Parm}} & 
   \multicolumn{1}{|S{header}{r}|}{\pbox[b]{\textwidth}{Estimate}}
\\\hline
\endhead
   \multicolumn{1}{|S{data}{l}|}{\pbox[b]{\textwidth}{Site}} & 
   \multicolumn{1}{|S{data}{r}|}{\pbox[b]{\textwidth}{ 14.6854}}
\\\hline
   \multicolumn{1}{|S{data}{l}|}{\pbox[b]{\textwidth}{Period*Site}} & 
   \multicolumn{1}{|S{data}{r}|}{\pbox[b]{\textwidth}{  1.6775}}
\\\hline
   \multicolumn{1}{|S{data}{l}|}{\pbox[b]{\textwidth}{Residual}} & 
   \multicolumn{1}{|S{data}{r}|}{\pbox[b]{\textwidth}{ 10.9274}}
\\\hline
\end{sastable}


\end{frame}


%******************************************************************************************************************
\begin{frame}
\frametitle{BACI Design 2}

{\it R} Analysis: (Be sure that Period, Site, and Siteclass are FACTORS).\\

\hspace{5mm} result <- lme(Density $\sim$ SiteClass+Period+SiteClass:Period,\\
\hspace{15mm} 	              data=blah, \\
\hspace{15mm} 	              random=$\sim$ 1 | Site / Period)\\
\hspace{5mm} anova(result)  \# Get the ANOVA Table\\
\hspace{5mm} summary(result) \# Get the BACI Effect\\
\hspace{5mm} VarCorr(result) \# Get the variance components\\

\vspace{2mm}
CAUTION: {\it R} does NOT have a KR adjustment so results may differ from {\it SAS}.\\
CAUTION: {\it R} {\it lmer()} function can also be used, but doesn't give p-values directly.

\end{frame}


%******************************************************************************************************************
\begin{frame}[fragile]  % must declare fragile if using \verbatim in the slide
\frametitle{BACI Design 2}

{\it R} results:\\[-4mm]
\verbatiminput{Images/baci-crabs-mod-R-300-type3.txt}
Note that test results differ (slightly) from {\it SAS}

\vspace{2mm} Estimate of BACI effect is:
\verbatiminput{Images/baci-crabs-mod-R-300baci.txt}

\end{frame}

%******************************************************************************************************************
\begin{frame}[fragile]  % must declare fragile if using \verbatim in the slide
\frametitle{BACI Design 2}

{\it R} results (continued):\\
\vspace{2mm} 
Estimate of Variance Components
\verbatiminput{Images/baci-crabs-mod-R-300-vc.txt}
\vspace{2mm}
CAUTION: {\it R} labels the results poorly!

\end{frame}



%******************************************************************************************************************
\begin{frame}
\frametitle{BACI Design 2}
Assumptions:
\begin{itemize}
\item Not necessary for design to be balanced, i.e.\ number of quadrats can vary among sites and years; not all sites need both before
and after information.
\item All measurements within and across sites and years at a site are INDEPENDENT of each other.
\item All sites are independent of each other.
\item Normality of residuals (but fairly robust IF....)
\item Equality of variation each each Site $\times$ Period
\item Normality of site effects; normality of {\it site*period} interactions. Difficult to assess because typically have a few sites.
\end{itemize}

\end{frame}

%******************************************************************************************************************
\begin{frame}
\frametitle{BACI Design 2}

\vspace{2mm}
Key Limitations and Difference from Design 1:
\begin{itemize}
\item Inference is now more general to all SITES, not just the 2 particular sites chosen in Design 1. Consequently, there can be
``loss of power'' but you are comparing two different inferences!
\item Limiting feature is the {\it site*period} variance component as you cannot influence by more sampling! Consequently, there is benefit
to sampling more site and/or more quadrats but you need to look at tradeoffs.
\end{itemize}

\end{frame}


%******************************************************************************************************************
\begin{frame}
\frametitle{BACI Design 2}

\includegraphics[height=0.5 \textheight]{Images/BACI-design-2-multicontrols.png}

Power (see web site for programs) depends on:
\begin{itemize}
\item Size of BACI effect $=(\mu_{ca}-\mu_{cb}) - (\mu_{ta}-\mu_{tb})$
\item TWO sources of variation important for planning:
\begin{itemize}
\item $\sigma_{quadrat}$ of replicates at each site-year combination
\item $\sigma_{site*year}$ 
\end{itemize}
$\sigma_{site}$ ``cancels' because each site is measured both before and after (A GOOD THING TO DO)!
\item Number of quadrats, number of sites.
\end{itemize}

\end{frame}

%******************************************************************************************************************
\begin{frame}
\frametitle{BACI Design 2}

Crabs example: $BACI=-5$, $\sigma_{quadrats}=3.30$, $\sigma_{site*year}=1.296$ 

\verbatiminput{Images/baci-crabs-mod-power-R-500.txt}

\vspace{2mm}
Aim for about 80\% power at $\alpha=0.05$.\\
\vspace{2mm}
Usually, more sites are preferable to more quadrats/site to improve power.
\end{frame}





%%%%%%%%%%%%%%%%%%%%%%%%%%%%%%%%%%%%%

%******************************************************************************************************************
\begin{frame}
\frametitle{BACI Design 3}
\includegraphics[height=0.7 \textheight]{Images/BACI-design-3-multiyear}

E.g.\ Fish counts (minnow traps) measured in control and impacted stream several years before and after impact. Only one measurement taken
per year per stream (this has implications later).
\end{frame}


%******************************************************************************************************************
\begin{frame}
\frametitle{BACI Design 3}
\includegraphics[height=0.4 \textheight]{Images/BACI-design-3-multiyear-varcomp.png}

Three levels of variation: 
\begin{itemize}
\item  quadrats-within-a-beach; These are pseudo-replicates
\item year-to-year. External factors that affect all sites the same way. Because measurements are paired within years, this variance component will ``cancel'' in the analysis. 
\item site*year interaction.  Measures the inconsistency of the response of sites to the temporal effects. 
\end{itemize}
Note that if you only have one quadrat, then quadrat and site*year variation is confounded and cannot be separated.
\end{frame}

%******************************************************************************************************************
\begin{frame}
\frametitle{BACI Design 3}
\includegraphics[height=0.4 \textheight]{Images/BACI-design-3-multiyear-varcomp.png}

What can you control via modifications to the sampling design?
\begin{itemize}
\item Quadrat variation can be controlled by choosing larger quadrats.
\item Year variation cannot be controlled, but by pairing the effects ``cancel''.
\item Site-Year variation cannot be controlled by modifying design. This is the limiting variation for the design.
\end{itemize}
\end{frame}


%******************************************************************************************************************
\begin{frame}
\frametitle{BACI Design 3}
Multiple (equivalent) ways to analyze this data:
\begin{itemize}
\item Find difference of means for EACH YEAR; analyze the differences in means using $t$-test to see if the mean difference before is the same as the mean difference after.. [Only approximate analysis if design is unbalanced.]
\item Mixed Effects ANOVA on the means of each site-year combination.
\item Mixed Effects ANOVA on the raw data. This is the most general and provides estimates of ALL variance components needed for power analysis.
\end{itemize}

\end{frame}


%******************************************************************************************************************
\begin{frame}
\frametitle{BACI Design 3}
Two-factor mixed-effect ANOVA.\\

{\it SAS} Analysis:\\
\hspace{5mm} PROC Mixed data=blah;\\
\hspace{5mm} CLASS Period Siteclass Year Site;\\
\hspace{5mm} MODEL Y = Period~~ Siteclass~~ Period*Siteclass / ddfm=kr;\\
\hspace{5mm} RANDOM Year(Period) year(Period)*Site;\\
\hspace{5mm} ESTIMATE 'BACI Effect' Period*Siteclass 1 -1 -1 1;\\

\vspace{3mm}
The test of no impact is the test for no interaction, i.e.\ test of $Period*Siteclass$ interaction.\\
Estimate BACI effect as DIFFERENTIAL change in means between two sites, i.e.\\[-4mm]
$$BACI=(\mu_{ca}-\mu_{cb}) - (\mu_{ta}-\mu_{tb})$$
\end{frame}


%******************************************************************************************************************
\begin{frame}
\frametitle{BACI Design 3}
Two-factor mixed-effect ANOVA.\\

\vspace{3mm}
{\it SAS} results (25 y; one measurement/year-site combination):\\[-4mm]
\input{Images/baci-fish-SAS-300-type3.tex}

\sascontents[1]{The Print Procedure}
\sascontents[2]{Data Set WORK.MIXED2ESTS}


\begin{sastable}[c]{lrrrrrrrr}\hline
   \multicolumn{1}{|S{header}{l}|}{\pbox[b]{\textwidth}{Label}} & 
   \multicolumn{1}{|S{header}{r}|}{\pbox[b]{\textwidth}{Estimate}} & 
   \multicolumn{1}{|S{header}{r}|}{\pbox[b]{\textwidth}{Standard \\ Error}} & 
   \multicolumn{1}{|S{header}{r}|}{\pbox[b]{\textwidth}{DF}} & 
   \multicolumn{1}{|S{header}{r}|}{\pbox[b]{\textwidth}{t \\ Value}} & 
   \multicolumn{1}{|S{header}{r}|}{\pbox[b]{\textwidth}{Pr \\ > \\ |t|}} & 
   \multicolumn{1}{|S{header}{r}|}{\pbox[b]{\textwidth}{Alpha}} & 
   \multicolumn{1}{|S{header}{r}|}{\pbox[b]{\textwidth}{Lower}} & 
   \multicolumn{1}{|S{header}{r}|}{\pbox[b]{\textwidth}{Upper}}
\\\hline
\endhead
   \multicolumn{1}{|S{data}{l}|}{\pbox[b]{\textwidth}{BACI effect}} & 
   \multicolumn{1}{|S{data}{r}|}{\pbox[b]{\textwidth}{ -8.9936}} & 
   \multicolumn{1}{|S{data}{r}|}{\pbox[b]{\textwidth}{ 15.3280}} & 
   \multicolumn{1}{|S{data}{r}|}{\pbox[b]{\textwidth}{  23}} & 
   \multicolumn{1}{|S{data}{r}|}{\pbox[b]{\textwidth}{  -0.59}} & 
   \multicolumn{1}{|S{data}{r}|}{\pbox[b]{\textwidth}{0.5631}} & 
   \multicolumn{1}{|S{data}{r}|}{\pbox[b]{\textwidth}{  0.05}} & 
   \multicolumn{1}{|S{data}{r}|}{\pbox[b]{\textwidth}{-40.7020}} & 
   \multicolumn{1}{|S{data}{r}|}{\pbox[b]{\textwidth}{ 22.7149}}
\\\hline
\end{sastable}




\end{frame}


%******************************************************************************************************************
\begin{frame}
\frametitle{BACI Design 3}
Two-factor mixed-effect ANOVA.\\

\vspace{3mm}
{\it SAS} results (continued)\\[-4mm]

\sascontents[1]{The Print Procedure}
\sascontents[2]{Data Set WORK.MIXED2COVPARMS}


\begin{sastable}[c]{lr}\hline
   \multicolumn{1}{|S{header}{l}|}{\pbox[b]{\textwidth}{Cov \\ Parm}} & 
   \multicolumn{1}{|S{header}{r}|}{\pbox[b]{\textwidth}{Estimate}}
\\\hline
\endhead
   \multicolumn{1}{|S{data}{l}|}{\pbox[b]{\textwidth}{SampleTime}} & 
   \multicolumn{1}{|S{data}{r}|}{\pbox[b]{\textwidth}{ 1005.95}}
\\\hline
   \multicolumn{1}{|S{data}{l}|}{\pbox[b]{\textwidth}{Residual}} & 
   \multicolumn{1}{|S{data}{r}|}{\pbox[b]{\textwidth}{  733.04}}
\\\hline
\end{sastable}


Note that if you only have one quadrat, then quadrat and site*year variation is confounded and cannot be separated.
So the residual variation represents both effects combined.

\end{frame}



%******************************************************************************************************************
\begin{frame}
\frametitle{BACI Design 3}

{\it R} Analysis: (Be sure that Period, {\bf Year --- common error}, and Siteclass are FACTORS).\\

\hspace{5mm} result <- lme(Density $\sim$ SiteClass+Period+SiteClass:Period,\\
\hspace{15mm} 	              data=blah, \\
\hspace{15mm} 	              random=$\sim$ 1 | Year / Site)\\
\hspace{5mm} anova(result)  \# Get the ANOVA Table\\
\hspace{5mm} summary(result) \# Get the BACI Effect\\
\hspace{5mm} VarCorr(result) \# Get the variance components\\

\vspace{2mm}
CAUTION: {\it R} does NOT have a KR adjustment so results may differ from {\it SAS}.\\
CAUTION: {\it R} {\it lmer()} function can also be used, but doesn't give p-values directly.

\end{frame}


%******************************************************************************************************************
\begin{frame}[fragile]  % must declare fragile if using \verbatim in the slide
\frametitle{BACI Design 3}

{\it R} results:\\[-4mm]
\verbatiminput{Images/baci-fish-R-300-type3.txt}

Note that test results differ (slightly) from {\it SAS}

\vspace{2mm} Estimate of BACI effect is:
\verbatiminput{Images/baci-fish-R-300baci.txt}


\end{frame}

%******************************************************************************************************************
\begin{frame}[fragile]  % must declare fragile if using \verbatim in the slide
\frametitle{BACI Design 3}

{\it R} results (continued):\\
\vspace{2mm} 
Estimate of Variance Components
\verbatiminput{Images/baci-fish-R-300-vc.txt}

\vspace{2mm}
CAUTION: {\it R} labels the results poorly!\\
Note that if you only have one quadrat, then quadrat and site*year variation is confounded and cannot be separated.
So the residual variation represents both effects combined.

\end{frame}



%******************************************************************************************************************
\begin{frame}
\frametitle{BACI Design 3}
Assumptions:
\begin{itemize}
\item Not necessary for design to be balanced, i.e.\ number of quadrats can vary among sites and years; not all sites need be measured in all years.
\item All measurements within and across sites and years at a site are INDEPENDENT of each other.
\item All sites are independent of each other.
\item Normality of residuals (but fairly robust IF....)
\item Equality of variation each each Site $\times$ Year
\item Normality of year effects; normality of {\it site*year} interactions. Difficult to assess because typically have a few years.
\end{itemize}

\end{frame}

%******************************************************************************************************************
\begin{frame}
\frametitle{BACI Design 3}

\vspace{2mm}
Key Limitations and Difference from Design 1:
\begin{itemize}
\item Inference is limited to these two sites, but now over ALL years. 
\item Note that a STEP CHANGE in the mean is assumed (see previous slides)!
\item Limiting feature is the {\it site*year} variance component as you cannot influence it by better sample design!
\item There is benefit to sampling more years and/or more quadrats, but you need to look at tradeoffs.
\end{itemize}

\end{frame}


%******************************************************************************************************************
\begin{frame}
\frametitle{BACI Design 3}

\includegraphics[height=0.5 \textheight]{Images/BACI-design-3-multiyear-varcomp}

Power (see web site for programs) depends on:
\begin{itemize}
\item Size of BACI effect $=(\mu_{ca}-\mu_{cb}) - (\mu_{ta}-\mu_{tb})$
\item TWO sources of variation important for planning:
\begin{itemize}
\item $\sigma_{quadrat}$ of replicates at each site-year combination
\item $\sigma_{site*year}$ 
\end{itemize}
$\sigma_{year}$ ``cancels' because each site is measured in all years (A GOOD THING TO DO)!
\item Number of quadrats, number of years before and after impact.
\end{itemize}

\end{frame}

%******************************************************************************************************************
\begin{frame}
\frametitle{BACI Design 3}

Fish counts example: $BACI=various$, $\sigma_{quadrats+site*year}=27.07$. [With only 1 measurement/year, you cannot
separate the two sources of variation.]

\verbatiminput{Images/baci-fish-power-R-500.txt}

\vspace{2mm}
Aim for about 80\% power at $\alpha=0.05$.\\
\vspace{2mm}
Usually, more years are preferable to more quadrats/site to improve power.
\end{frame}




%%%%%%%%%%%%%%%%%%%%%%%%%%%%%%%%%%%%%

%******************************************************************************************************************
\begin{frame}
\frametitle{BACI Design 4}
\includegraphics[height=0.7 \textheight]{Images/BACI-design-4-multiyearmultisite}

E.g.\ Fry counts (minnow traps) measured in several control and impacted streams several years before and after impact. 
Multiple traps per site-year combination.
\end{frame}


%******************************************************************************************************************
\begin{frame}
\frametitle{BACI Design 4}
\includegraphics[height=0.35 \textheight]{Images/BACI-design-4-multiyearmultisite-varcomp.png}

Four levels of variation: 
\begin{itemize}
\item  trap-to-trap; These are pseudo-replicates
\item year-to-year. External factors that affect all sites the same way. Because measurements are paired within years, this variance component will ``cancel'' in the analysis. 
\item site-to-site. Sites are naturally different. Because all sites measured before and after, this variance component will
``cancel'' in the analysis.
\item site*year interaction.  Measures the inconsistency of the response of sites to the temporal effects. 
\end{itemize}
\end{frame}

%******************************************************************************************************************
\begin{frame}
\frametitle{BACI Design 4}
\includegraphics[height=0.4 \textheight]{Images/BACI-design-4-multiyearmultisite-varcomp.png}

What can you control via modifications to the sampling design?
\begin{itemize}
\item Trap variation can be controlled by soaking longer or using larger traps.
\item Year variation cannot be controlled, but by pairing the effects ``cancel''.
\item Site variation cannot be controlled, but by measuring same site over time, the effect ``cancel''.
\item Site-Year variation cannot be controlled by modifying design. This is the limiting variation for the design.
\end{itemize}
\end{frame}


%******************************************************************************************************************
\begin{frame}
\frametitle{BACI Design 4}

No simple way to analyze this design except via a fully specified mixed-effects model!

\end{frame}


%******************************************************************************************************************
\begin{frame}
\frametitle{BACI Design 4}
Two-factor mixed-effect ANOVA.\\

{\it SAS} Analysis:\\
\hspace{5mm} PROC Mixed data=blah nobound;\\
\hspace{5mm} CLASS Period Siteclass Site Year;\\
\hspace{5mm} MODEL Y = Period~~ Siteclass~~ Period*Siteclass / ddfm=kr;\\
\hspace{5mm} RANDOM Year(Period) site(SiteClass)\\
\hspace{15mm}               year(Period)*site(SiteClass);\\
\hspace{5mm} ESTIMATE 'BACI Effect' Period*Siteclass 1 -1 -1 1;\\

\vspace{3mm}
The test of no impact is the test for no interaction, i.e.\ test of $Period*Siteclass$ interaction.\\
Estimate BACI effect as DIFFERENTIAL change in means between two sites, i.e.\\[-4mm]
$$BACI=(\mu_{ca}-\mu_{cb}) - (\mu_{ta}-\mu_{tb})$$
\end{frame}


%******************************************************************************************************************
\begin{frame}
\frametitle{BACI Design 4}
Two-factor mixed-effect ANOVA.\\

\vspace{3mm}
{\it SAS} results:\\[-4mm]
\sascontents[1]{The Mixed Procedure}
\sascontents[2]{Type 3 Tests of Fixed Effects}


\begin{sastable}[c]{lrrrr}\hline
   \multicolumn{5}{|S{header}{c}|}{\pbox[b]{\textwidth}{Type 3 Tests of Fixed Effects}}
\\\hline
   \multicolumn{1}{|S{header}{l}|}{\pbox[b]{\textwidth}{Effect}} & 
   \multicolumn{1}{|S{header}{r}|}{\pbox[b]{\textwidth}{Num DF}} & 
   \multicolumn{1}{|S{header}{r}|}{\pbox[b]{\textwidth}{Den DF}} & 
   \multicolumn{1}{|S{header}{r}|}{\pbox[b]{\textwidth}{F Value}} & 
   \multicolumn{1}{|S{header}{r}|}{\pbox[b]{\textwidth}{Pr~>~F}}
\\\hline
\endhead
   \multicolumn{1}{|S{rowheader}{l}|}{\pbox[b]{\textwidth}{SiteClass}} & 
   \multicolumn{1}{|S{data}{r}|}{\pbox[b]{\textwidth}{1}} & 
   \multicolumn{1}{|S{data}{r}|}{\pbox[b]{\textwidth}{23}} & 
   \multicolumn{1}{|S{data}{r}|}{\pbox[b]{\textwidth}{2.28}} & 
   \multicolumn{1}{|S{data}{r}|}{\pbox[b]{\textwidth}{0.1444}}
\\\hline
   \multicolumn{1}{|S{rowheader}{l}|}{\pbox[b]{\textwidth}{Period}} & 
   \multicolumn{1}{|S{data}{r}|}{\pbox[b]{\textwidth}{1}} & 
   \multicolumn{1}{|S{data}{r}|}{\pbox[b]{\textwidth}{23}} & 
   \multicolumn{1}{|S{data}{r}|}{\pbox[b]{\textwidth}{0.93}} & 
   \multicolumn{1}{|S{data}{r}|}{\pbox[b]{\textwidth}{0.3438}}
\\\hline
   \multicolumn{1}{|S{rowheader}{l}|}{\pbox[b]{\textwidth}{SiteClass*Period}} & 
   \multicolumn{1}{|S{data}{r}|}{\pbox[b]{\textwidth}{1}} & 
   \multicolumn{1}{|S{data}{r}|}{\pbox[b]{\textwidth}{23}} & 
   \multicolumn{1}{|S{data}{r}|}{\pbox[b]{\textwidth}{0.34}} & 
   \multicolumn{1}{|S{data}{r}|}{\pbox[b]{\textwidth}{0.5631}}
\\\hline
\end{sastable}



\sascontents[1]{The Print Procedure}
\sascontents[2]{Data Set WORK.MIXED2ESTS}


\begin{sastable}[c]{lrrrrrrrr}\hline
   \multicolumn{1}{|S{header}{l}|}{\pbox[b]{\textwidth}{Label}} & 
   \multicolumn{1}{|S{header}{r}|}{\pbox[b]{\textwidth}{Estimate}} & 
   \multicolumn{1}{|S{header}{r}|}{\pbox[b]{\textwidth}{Standard \\ Error}} & 
   \multicolumn{1}{|S{header}{r}|}{\pbox[b]{\textwidth}{DF}} & 
   \multicolumn{1}{|S{header}{r}|}{\pbox[b]{\textwidth}{t \\ Value}} & 
   \multicolumn{1}{|S{header}{r}|}{\pbox[b]{\textwidth}{Pr \\ > \\ |t|}} & 
   \multicolumn{1}{|S{header}{r}|}{\pbox[b]{\textwidth}{Alpha}} & 
   \multicolumn{1}{|S{header}{r}|}{\pbox[b]{\textwidth}{Lower}} & 
   \multicolumn{1}{|S{header}{r}|}{\pbox[b]{\textwidth}{Upper}}
\\\hline
\endhead
   \multicolumn{1}{|S{data}{l}|}{\pbox[b]{\textwidth}{baci contrast}} & 
   \multicolumn{1}{|S{data}{r}|}{\pbox[b]{\textwidth}{  1.0512}} & 
   \multicolumn{1}{|S{data}{r}|}{\pbox[b]{\textwidth}{  2.5022}} & 
   \multicolumn{1}{|S{data}{r}|}{\pbox[b]{\textwidth}{   1}} & 
   \multicolumn{1}{|S{data}{r}|}{\pbox[b]{\textwidth}{   0.42}} & 
   \multicolumn{1}{|S{data}{r}|}{\pbox[b]{\textwidth}{0.7468}} & 
   \multicolumn{1}{|S{data}{r}|}{\pbox[b]{\textwidth}{  0.05}} & 
   \multicolumn{1}{|S{data}{r}|}{\pbox[b]{\textwidth}{-30.7417}} & 
   \multicolumn{1}{|S{data}{r}|}{\pbox[b]{\textwidth}{ 32.8441}}
\\\hline
\end{sastable}




\end{frame}


%******************************************************************************************************************
\begin{frame}
\frametitle{BACI Design 4}
Two-factor mixed-effect ANOVA.\\

\vspace{3mm}
{\it SAS} results (continued):\\[-4mm]

\sascontents[1]{The Print Procedure}
\sascontents[2]{Data Set WORK.MIXED2COVPARMS}


\begin{sastable}[c]{lr}\hline
   \multicolumn{1}{|S{header}{l}|}{\pbox[b]{\textwidth}{Cov \\ Parm}} & 
   \multicolumn{1}{|S{header}{r}|}{\pbox[b]{\textwidth}{Estimate}}
\\\hline
\endhead
   \multicolumn{1}{|S{data}{l}|}{\pbox[b]{\textwidth}{Year(Period)}} & 
   \multicolumn{1}{|S{data}{r}|}{\pbox[b]{\textwidth}{  0.1239}}
\\\hline
   \multicolumn{1}{|S{data}{l}|}{\pbox[b]{\textwidth}{Site(SiteClass)}} & 
   \multicolumn{1}{|S{data}{r}|}{\pbox[b]{\textwidth}{  0.2963}}
\\\hline
   \multicolumn{1}{|S{data}{l}|}{\pbox[b]{\textwidth}{Year*Site(Site*Peri)}} & 
   \multicolumn{1}{|S{data}{r}|}{\pbox[b]{\textwidth}{ -4.3630}}
\\\hline
   \multicolumn{1}{|S{data}{l}|}{\pbox[b]{\textwidth}{Residual}} & 
   \multicolumn{1}{|S{data}{r}|}{\pbox[b]{\textwidth}{  4.3331}}
\\\hline
\end{sastable}



A negative estimate of variance is possible (and just means that the variance component is small, i.e.\ close to zero) which is nice
for this example.

\end{frame}



%******************************************************************************************************************
\begin{frame}
\frametitle{BACI Design 4}

{\it R} Analysis: (Be sure that Period, {\bf Year --- common error}, Site, and Siteclass are FACTORS).\\
\vspace{2mm}
{\Large NOT FOR THE FAINT OF HEART!}\\
\vspace{2mm}
\hspace{5mm} SOME R MAGIC - see my website.\\
\hspace{5mm} anova(result)  \# Get the ANOVA Table\\
\hspace{5mm} summary(result) \# Get the BACI Effect\\
\hspace{5mm} VarCorr(result) \# Get the variance components\\

\vspace{2mm}
CAUTION: {\it R} does NOT have a KR adjustment so results may differ from {\it SAS}.\\
CAUTION: {\it R} {\it lmer()} function can also be used, but doesn't give p-values directly.

\end{frame}


%******************************************************************************************************************
\begin{frame}[fragile]  % must declare fragile if using \verbatim in the slide
\frametitle{BACI Design 4}

{\it R} results:\\[-4mm]
\verbatiminput{Images/baci-fry-R-300-type3.txt}

Note that test results differ (considerably) from {\it SAS} but the final conclusion is similar. 

\vspace{2mm} Estimate of BACI effect is:
\verbatiminput{Images/baci-fry-R-300baci.txt}


\end{frame}

%******************************************************************************************************************
\begin{frame}[fragile]  % must declare fragile if using \verbatim in the slide
\frametitle{BACI Design 4}

{\it R} results (continued):\\
\vspace{2mm} 
Estimate of Variance Components
\verbatiminput{Images/baci-fry-R-300-vc.txt}

\vspace{2mm}
CAUTION: {\it R} labels the results poorly!\\

\end{frame}



%******************************************************************************************************************
\begin{frame}
\frametitle{BACI Design 4}
Assumptions:
\begin{itemize}
\item Not necessary for design to be balanced, i.e.\ number of traps can vary among sites and years; not all sites need be measured in all years.
\item All measurements within and across sites and years at a site are INDEPENDENT of each other.
\item All sites are independent of each other.
\item Normality of residuals (but fairly robust IF....)
\item Equality of variation each each Site $\times$ Year
\item Normality of year effects; normality of site effects; normality of {\it site*year} interactions. Difficult to assess because typically have a few years.
\end{itemize}

\end{frame}

%******************************************************************************************************************
\begin{frame}
\frametitle{BACI Design 4}

\vspace{2mm}
Key Limitations and Difference from Design 1:
\begin{itemize}
\item Inference generalized to ALL sites and over ALL years. 
\item Note that a STEP CHANGE in the mean is assumed (see previous slides)!
\item Limiting feature is the {\it site*year} variance component as you cannot influence it by better sample design!
\item There is benefit to sampling more years and/or more quadrats, but you need to look at tradeoffs.
\end{itemize}

\end{frame}


%******************************************************************************************************************
\begin{frame}
\frametitle{BACI Design 4}

\includegraphics[height=0.4 \textheight]{Images/BACI-design-4-multiyearmultisite-varcomp}

Power (see web site for programs) depends on:
\begin{itemize}
\item Size of BACI effect $=(\mu_{ca}-\mu_{cb}) - (\mu_{ta}-\mu_{tb})=5$
\item TWO sources of variation important for planning:
\begin{itemize}
\item $\sigma_{quadrat}$ of replicates at each site-year combination
\item $\sigma_{site*year}$ 
\end{itemize}
Other effects (year and site) ``cancel'' because of pairing and blocking.
\item Number of quadrats, number of years before and after impact, number of sites impact and control.
\end{itemize}

\end{frame}

%******************************************************************************************************************
\begin{frame}
\frametitle{BACI Design 4}

Fish counts example: $BACI=5$, $\sigma_{quadrats}=0.75$, $\sigma_{site*year}=0.1$. 

{\tiny
\verbatiminput{Images/baci-fry-power-R-500.txt}
}
\vspace{2mm}
Aim for about 80\% power at $\alpha=0.05$.\\
\end{frame}

%%%%%%%%%%%%%%%%%%%%%%%%%%%%%%%%%%%%%%%%%%%%%%%%
%%%%%%%%%%%%%%%%%%%%%%%%%%%%%%%%%%%%%%%%%%%%%%%%
%%%%%%%%%%%%%%%%%%%%%%%%%%%%%%%%%%%%%%%%%%%%%%%%
%%%%%%%%%%%%%%%%%%%%%%%%%%%%%%%%%%%%%%%%%%%%%%%%
%%%%%%%%%%%%%%%%%%%%%%%%%%%%%%%%%%%%%%%%%%%%%%%%




%******************************************************************************************************************
\begin{frame}
\frametitle{Beyond BACI}

What do you do for long-term studies and/or no/few pre-impact measurements?
\vspace{5mm}

Wiens, J.A., Parker, K. R. (1995).\\
Analyzing the Effects of Accidental Environmental Impacts: Approaches and Assumptions. \\
Ecological Applications  5,  1069-1083. \\
\url{http://dx.doi.org/10.2307/2269355}.
\end{frame}


%******************************************************************************************************************
\begin{frame}
\frametitle{Beyond BACI - Spatial Impact-Reference design}

Sampling is performed immediately
after impact at two sites - the impacted and non-impacted sites.
\vspace{3mm}
\includegraphics[width=\textwidth]{Images/AccidentalImpacts/acc-imp-d1}

\end{frame}

%******************************************************************************************************************
\begin{frame}
\frametitle{Beyond BACI - Spatial Regression design}

Sampling is performed at a number of sites 
over the range of exposure (e.g.\ by the amount of oil washed ashore). A regression
of abundance against the exposure is drawn.
\vspace{3mm}
\includegraphics[width=\textwidth]{Images/AccidentalImpacts/acc-imp-d2}

\end{frame}

%******************************************************************************************************************
\begin{frame}
\frametitle{Beyond BACI - Spatial Matched-pair design}

Sampling is done on randomly selected
impact sites and control sites that are matched on relevant natural factors, e.g. type 
of substrate where the shell fish aggregate.

\vspace{3mm}
\includegraphics[width=\textwidth]{Images/AccidentalImpacts/acc-imp-d3}

\end{frame}

%******************************************************************************************************************
\begin{frame}
\frametitle{Beyond BACI - Temporal Baseline design}

Sometimes, fortuitous surveys have been
done at the same site before the impact occurred. Sampling takes place at 
the same site after the impact.

\vspace{3mm}
\includegraphics[width=\textwidth]{Images/AccidentalImpacts/acc-imp-d4}

\end{frame}

%******************************************************************************************************************
\begin{frame}
\frametitle{Beyond BACI - Temporal Time-series design}

The impacted site is surveyed repeated over
a long period of time (e.g.\ bi-monthly for 2 years) and the results plotted.

\vspace{3mm}
\includegraphics[width=\textwidth]{Images/AccidentalImpacts/acc-imp-d5}

\end{frame}

%******************************************************************************************************************
\begin{frame}
\frametitle{Beyond BACI - Temporal-Spatial Pre-post design}

Similar to the classical BACI design except
that pre/post samples are taken at sites that vary in the degree of exposure to 
the impact.

\vspace{3mm}
\includegraphics[width=\textwidth]{Images/AccidentalImpacts/acc-imp-d6}

\end{frame}

%******************************************************************************************************************
\begin{frame}
\frametitle{Beyond BACI - Temporal-Spatial Level-by-time design}

The impact site is measured  over time
from the time of impact. A control site is also measured over time from the time of impact 
at the same sampling occasions.

\vspace{3mm}
\includegraphics[width=\textwidth]{Images/AccidentalImpacts/acc-imp-d7}

\end{frame}

%******************************************************************************************************************
\begin{frame}
\frametitle{Beyond BACI - Temporal-Spatial Impact trend-by-time design}

The Regression design is performed 
at the impacted site just after impact. The same design is performed at the impacted site
a year or longer after impact. Both plots of response vs.\ dose are plotted on the same graph.

\vspace{3mm}
\includegraphics[width=\textwidth]{Images/AccidentalImpacts/acc-imp-d8}

\end{frame}

%******************************************************************************************************************
\begin{frame}
\frametitle{Beyond BACI - Key Assumptions}

Key assumptions for SPATIAL designs.
\begin{itemize}
\item Equal natural factors at impacted and non-impacted areas. However, 
because contamination was not randomized, there is no guarantee
of equal natural factors.
\item Sampling interval is short relative to temporal variation. This 
guarantees that the measurements show the effect
of the impact and not just difference that would have occurred
naturally over time.
\end{itemize}
\end{frame}

%******************************************************************************************************************
\begin{frame}
\frametitle{Beyond BACI - Key Assumptions}

Key assumptions for TEMPORAL designs.
\begin{itemize}
\item Natural factors are in steady-state equilibrium, i.e. the population
levels remain the same over time in the absence of an impact.
\item Differences in sampling personnel and sampling methods over time
are inconsequential so that the observed differences are related 
to the impact and not to differences in methods.
\end{itemize}

\end{frame}

%******************************************************************************************************************
\begin{frame}
\frametitle{Beyond BACI - Key Assumptions}
Key assumptions for SPATIAL-TEMPORAL designs.
\begin{itemize}
\item  Natural factors and the biological resource are in dynamic equilibrium
among area. The level of a resource changes similarly for different
areas -- responding similarly to changing climatic conditions and populations. 
\item Consistent sampling methods over time.
\end{itemize}
\end{frame}


%******************************************************************************************************************
\begin{frame}
\frametitle{Beyond BACI - Ranking Designs}

Ability of initial impact and recovery?
\begin{tabular}{lp{1.0in}p{1.0in}}
{\bf Design} & {\bf Assessing initial impact} & {\bf Assessing recovery}  \\  \hline
Spatial Impact-reference & x &          \\
Spatial Regression       & x &          \\
Spatial Matched pairs    & x &          \\
Temporal Baseline        & x &          \\
Temporal Time series     & x &  x       \\
ST Pre-post pairs        & x &          \\
ST Level-by-time         & x &  x       \\
ST Trend-by-time         & x &  x       \\
\end{tabular}
\end{frame}

%******************************************************************************************************************
\begin{frame}
\frametitle{Beyond BACI - Defensibility}

Defensibility to determine if impact occurred (4=better)
\begin{tabular}{lc}
{\bf Design} & {\bf Ranking}  \\  \hline
Spatial Impact-reference &   2      \\
Spatial Regression       &   2      \\
Spatial Matched pairs    &   2      \\
Temporal Baseline        &   1      \\
Temporal Time series     &   2      \\
ST Pre-post pairs        &   3      \\
ST Level-by-time         &   4      \\
ST Trend-by-time         &   4      \\
\end{tabular}
\end{frame}

%******************************************************************************************************************
\begin{frame}
\frametitle{Summary}

Four standard BACI designs:
\begin{enumerate}
\item Single impact site; single control site; one year before; one year after.
\item Single/multiple impact site; multiple control sites; one year before; one year after.
\item Single impact site; single control site; multiple years before; multiple years after.
\item Single impact site; multiple control sites; multiple years before; multiple years after.
\end{enumerate}

\end{frame}

%******************************************************************************************************************
\begin{frame}
\frametitle{Summary}

Analysis of such designs
\begin{itemize}
\item For all but simple BACI (Design 1), mixed-effect ANOVA needed due to presence of multiple-sources of variation. [Some simplification
possible under special cases.]
\item Beware of pseudo-replication (quadrats vs.\ sites).
\item Single site/single year designs have NARROW inference space.
\item {\it Excel} - hopeless; {\it R} - you get what you pay for; {\it SAS} - best
\item See my web pages for examples.
\end{itemize}

Planning BACI designs
\begin{itemize}
\item Need estimates of quadrat-to-quadrat variance and site-year interaction variance.
\item Site effects and year effects ``cancel'' and can be ignored.
\item Usually better to do more sites and/or year than to measure more quadrats in each site-year combination.
\end{itemize}

\end{frame}

%******************************************************************************************************************
\begin{frame}
\frametitle{Summary}

\begin{itemize}
\item Weins and Parker (1995) paper if no or limited pre-impact measurements.
\end{itemize}
\vspace{5mm}
\begin{center}
DO NOT PANIC!\\
\vspace{3mm}
\url{http://www.stat.sfu.ca/~cschwarz/CourseNotes}\\
\vspace{3mm}
cschwarz @ stat.sfu.ca
\end{center}
\end{frame}


\end{document}